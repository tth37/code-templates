\documentclass[UTF8]{article}
\usepackage{amsmath}
\usepackage{xeCJK}
\usepackage{fancyhdr}
\usepackage{lastpage}
\usepackage{minted}
\usepackage{geometry}
\title{Untitled template codes}
\author{tth37,luckyyou,rabbyte}
\geometry{left=3.18cm,right=3.18cm,top=2.54cm,bottom=2.54cm}
\pagestyle{fancy}
\cfoot{第 \thepage 页 \qquad 共 \color{blue}{\pageref{LastPage}} \color{black} 页}
\begin{document}
    \maketitle
    \setcounter{page}{0}
    \thispagestyle{empty}
    \clearpage
    \tableofcontents
    \clearpage
    \section{数学}
        \subsection{数论}
            \subsubsection{扩展欧拉函数}
            $$
            a^b\equiv
            \begin{cases}
                a^{b\bmod\varphi(p)} & \gcd(a,p)=1\\
                a^b,\gcd(a,\,p)\ne1 & b<\varphi(p)\\
                a^{b\bmod\varphi(p)+\varphi(p)} & \gcd(a,\,p)\ne1,b\ge\varphi(p)
            \end{cases}
            \pmod p
            $$
            \subsubsection{线性筛与积性函数}
            \begin{minted}[mathescape]{cpp}
#include<cstring>
const int N = 1e6 + 10;
int phi[N], mu[N], primes[N], cnt = 0;
bool is_prime[N];
void linear_sieve(int n) {
    phi[1] = 1; mu[1] = 1;
    memset(is_prime, true, sizeof is_prime); is_prime[1] = false;
    for (int i = 2; i <= n; i++) {
        if (is_prime[i]) {
            primes[++cnt] = i;
            phi[i] = i - 1;
            mu[i] = -1;
        }
        for (int j = 1; j <= cnt && i * primes[j] <= n; j++) {
            is_prime[i * primes[j]] = false;
            if (i % primes[j]) {
                phi[i * primes[j]] = phi[i] * phi[primes[j]];
                mu[i * primes[j]] = -mu[i];
            }
            else {
                phi[i * primes[j]] = phi[i] * primes[j]; //$\varphi (p^n) = p^n - p^{n-1}$
                mu[i * primes[j]] = 0;
                break;
            }
        }
    }
}
            \end{minted}
\end{document}